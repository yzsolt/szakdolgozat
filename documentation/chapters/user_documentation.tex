
\section{A megoldott probléma áttekintése}

A szakdolgozat célja egy olyan program létrehozása volt, amelynek segítségével a valósidejű fizikai alapú megjelenítést, annak előnyeit be lehet mutatni. A fejlesztés eredménye egy C++ programnyelven írt, platformfüggetlen, OpenGL alapú valósidejű megjelenítő lett, amely egyszerű 3D-s modellek megjelenítésére képes, továbbá a modell környezetét ún. skybox-szal szimulálja és a modell anyagtulajdonságainak interaktív módosítására is lehetőséget ad.

\section{A felhasznált módszerek rövid leírása}

A program C++ programnyelven íródott és az OpenGL grafikus API-t használja. A megjelenítés során az alapértelmezett fizikai alapú mellett választható a korábban használt Blinn-Phong árnyalás~\cite{blinn1977models}. A fizikai alapú megjelenítés az árnyalási egyenlet~\cite{immel1986radiosity} megoldására épül, ehhez valósidőben több közelítés alkalmazása is szükséges: Cook-Torrance modell~\cite{cook1981reflectance} a spekuláris BRDF (kétirányú visszaverődési eloszlásfüggvény), Schlick közelítés~\cite{schlick1994inexpensive} a Fresnel-tényező kiszámolásához.

A nagy dinamikatartományú megjelenítéssel és az ehhez tartozó színleképezéssel lehetőség van a valóságban tapasztalható fényviszonyok, kontrasztarányok szimulálására az egyébként kis dinamikatartományú kijelzőkön. A fényviszonyokhoz történő adaptáció Erik Reinhard\cite{reinhard2002photographic} módszerét használja az átlagos fénysűrűség megállapításához, majd ebből a DICE prezentációja~\cite{dice_moving_frostbite_to_pbr} alapján számol expozíciós értéket.

A kép alapú megvilágításhoz használt részekre bontott integrál-közelítés és a hozzá tartozó árnyalóprogramok implementációja az Unreal 4 játékmotorban használt módszeren~\cite{karis2013real} alapul.

\section{Hardveres követelmények}

Minimális konfiguráció:

\begin{itemize}
  \item 64 bites, x86 architektúrájú processzor
  \item 128 MB memória
  \item legalább 50 MB szabad tárhely a háttértárolón
  \item OpenGL 3.3-at támogató videokártya
\end{itemize}

\noindent
Ajánlott konfiguráció:

\begin{itemize}
  \item legalább 2 GHz-es, 64 bites, x86 architektúrájú processzor
  \item 512 MB memória
  \item 300 MB szabad tárhely a háttértárolón
  \item OpenGL 3.3-at támogató, minimum középkategóriás videokártya
\end{itemize}

\section{Szoftveres követelmények}

A mellékelt bináris 64 bites Windows operációs rendszerhez készült, azonban a forráskód a három legnagyobb asztali operációs rendszer (Windows, Linux, OS X) bármelyikére lefordítható, 32 és 64 bites változatban is, ennek menetéről bővebben a következő fejezet ad felvilágosítást.

Mivel a megjelenítő OpenGL alapú, ezért a futtató számítógépen elérhetőnek kell lennie az OpenGL 3.3-as verziójának. Ezt a verziót mind az integrált, mind a dedikált videokártyák évekre visszamenőleg támogatják. Ha nem vagyunk biztosak a támogatott verzióban, akkor a melléklet \textit{bin} mappájában található glewinfo.exe binárist futtatva egy glewinfo.txt szöveges állomány jön létre, amelynek 8. sorában szerepel a grafikus kártya által támogatott OpenGL verzió (pl. ,,OpenGL version 4.3.0 - Build 20.19.15.4424 is supported'': a 4.3-as verzió támogatott).

\section{A program üzembe helyezése}

Ha 64 bites Windows rendszert használunk, akkor a mellékelt, előre lefordított bináris használható. Ha viszont más operációs rendszeren dolgozunk, vagy tovább szeretnénk fejleszteni a programot, akkor azt az adott platformra nekünk kell lefordítanunk. Ennek minimális feltétele a CMake keresztplatformos buildrendszer és valamilyen C/C++ fordító megléte a számítógépen.

A mellékelt forráskód könyvtár a végleges bináris elkészítésekor fennálló állapotot őrzi, azaz tartalmazza a függőségek Git tárolóit, forráskódjait és --- ahol szükséges --- a 64 bites Windows-ra lefordított Debug és Release bináris könyvtárfájlokat is az elkészítésükhöz használt Microsoft Visual Studio 2015 projektfájlokkal együtt, így egyező fejlesztői környezet esetén a továbbfejlesztés minden további teendő nélkül, azonnal elkezdhető.

A főprogram fordítása egyszerű: lépjünk az előzőleg létrehozott \textit{build} könyvtárba, majd adjuk ki a \textit{cmake ../} parancsot a megfelelő paraméterekkel, amelyekről a CMake program súgója (\textit{cmake -{}-help}) ad bővebb felvilágosítást. Amennyiben grafikus fejlesztőkörnyezetet szeretnénk használni, úgy a CMake számára a \textit{-G} kapcsolóval megadhatjuk, hogy milyen projektfájlt generáljon. Ezután a generált projektfájlt használhatjuk a program fordításához, azonban a projekt módosulása esetén (pl. forrásfájlok hozzáadása vagy törlése) a változtatásokat a CMakeLists.txt-ben meg kell adni, és a projektfájlt újra kell generáltatni.

Más operációs rendszer/processzor architektúra esetén a függőségek közül legalább a GLFW és NanoGUI könyvtárat újra le kell fordítanunk. Mindkettő a CMake build rendszert használja, így fordításuk hasonlóan történik a főprograméhoz. Kérdés esetén a projektek saját weboldalán találunk segítséget. Ügyeljünk rá, hogy a megfelelő \textit{glfw-build} és \textit{nanogui-build} mappából indítsuk a fordítást, mert a főprogram később ezekben fogja keresni az előre lefordított könyvtárfájlokat.

\section{A felhasználói felület áttekintése}

\begin{figure}[!ht]
    \centering
    \includegraphics[width=1.0\textwidth]{images/screenshot.png}
    \caption{A megjelenítő felhasználói felülete.}
\end{figure}

A megjelenítő felhasználói felülete négy ablakból áll, ezeken különféle információkat és beállítási lehetőségeket érhetünk el. A felhasználói felület nyelve angol, ezért alapszintű angol nyelvtudás szükséges a program használatához, de a következő alpontokban leírtak segítségével nyelvtudás nélkül is képesek leszünk legalább alapszinten használni a programot.

A felhasználói felület hátterében látjuk magát a betöltött modellt, illetve annak panorámaképből összeállított környezetét. Az egérgörgő segítségével változtathatjuk a virtuális kamera távolságát a modelltől, azaz így közelíthetünk vagy távolodhatunk tőle. A bal egérgomb lenyomásával és nyomva tartásával egy képzeletbeli gömbön mozgathatjuk a kamerát, ami mindig a modell közepe felé néz, így a modellt minden irányból megnézhetjük. A megjelenítő ablaka futás közben technikai okok miatt nem méretezhető át. Az alapértelmezett felbontás \(1280 \times 720\), de a következő, egymástól szóközzel elválasztott parancssori paraméterek segítségével elindítható a kívánt felbontáson:

\begin{itemize}[noitemsep]
\item \textbf{width}=szélesség (képpontban)
\item \textbf{height}=magasság (képpontban)
\end{itemize}

\noindent
További indítási paraméterek még:

\begin{itemize}[noitemsep]
\item \textbf{fullscreen}: ha megadjuk ezt a kulcsot, a megjelenítő teljes képernyős módban indul
\item \textbf{msaa}=MSAA élsimítási mintaszám (kettő hatványa)
\end{itemize}

A hardver által támogatott mintaszám túllépése esetén az elérhető legnagyobb mintaszámot fogja használni a megjelenítő. Az MSAA élsimítás költsége alapesetben is jelentős, ráadásul a mintaszám növelésével tovább nő, ezért csak megfelelően erős grafikus kártya mellett ajánlott a használata. Teljes képernyős módban a tallózó párbeszédpanelek (modell vagy skybox betöltés) a háttérbe kényszerítik a megjelenítő ablakát, emiatt a párbeszédpanel bezáródása után kézzel újra az előtérbe kell hozni azt (pl. Windows operációs rendszer esetén az Alt+TAB billentyűkombinációval). Továbbá teljes képernyős módban a megjelenítő nem veszi figyelembe a megadott felbontást, a képernyő natív felbontását használja.

\clearpage

\subsection{Megjelenítő beállításai}

\begin{wrapfigure}{l}{0.3\textwidth}
    \vspace{-23pt}
    \includegraphics[width=0.28\textwidth]{images/renderer_settings.png}
    \vspace{-20pt}
\end{wrapfigure}

A megjelenítő beállításai ablakban elsősorban a modelltől független, globális beállítások találhatóak. A V-Sync a vertikális szinkronizáció be- vagy kikapcsolására szolgál. Alapértelmezésben be van kapcsolva, így a megjelenítő mindig megvárja, amíg az elkészült képkocka megjelenítése befejeződik. Ezzel elkerülhető az ún. \textit{tearing}, ha a képernyő frissítési frekvenciájától (jellemzően 60 Hz) eltérő FPS (\textit{frames per second}, képkocka per másodperc) sebességgel dolgozik a megjelenítő. Kikapcsolásával a program az előző képkocka befejezése után egyből elkezdi a következő rajzolását, így ez a mód teljesítménymérésre is alkalmas lehet.

A \textit{Load skybox} ill. \textit{Load mesh} gombokkal egy fájlválasztó ablak nyílik meg, amelynek segítségével új skybox-ot vagy modellt tölthetünk be. Skybox esetén Radiance HDR (.hdr), modellek esetén Wavefront OBJ (.obj) a támogatott fájlformátum. Továbbá ügyeljünk arra, hogy a skybox-nak megadott kép vertikális kereszt alakban tartalmazza a kockatérkép oldalait. A mellékelt skybox-ok a \textit{data/skyboxes}, a modellek pedig a \textit{data/meshes} mappában találhatóak.

A \textit{Tone mapping} címke alatt található választógomb segítségével a beépített színleképezési operátorok (Uncharted 2, Reinhard, Unreal 4) közül választhatunk, ill. ki is kapcsolhatjuk a színleképezést az Off bejegyzés választásával.

Az \textit{Exposure} címke alatt lehetőségünk van az automatikus expozíciót be- vagy kikapcsolni. Ez az effektus az emberi szem eltérő fényerősséghez történő alkalmazkodását szimulálja, például amikor egy sötét szobából egy világos térbe lépünk ki. Kikapcsolt állapotában kézzel lehet állítani az expozíciót a lentebb található csúszka segítségével.

Az utolsó beállítás, a \textit{Rotate mesh} jelölőnégyzet segítségével a betöltött modell automatikus forgatását kapcsolhatjuk be/ki.

\clearpage

\subsection{Modell információ}

\begin{wrapfigure}{l}{0.3\textwidth}
    \vspace{-23pt}
    \includegraphics[width=0.28\textwidth]{images/mesh_info.png}
    \vspace{-20pt}
\end{wrapfigure}

A modell információs ablakban a modellel kapcsolatos információk és beállítások kaptak helyet. A \textit{Use PBR} kapcsolóval válthatunk az alapértelmezett fizikai alapú megjelenítés, illetve a hagyományos Blinn-Phong árnyalás között. Az alatta található választógombbal az OBJ fájlban található alakzatok (\textit{shape}-ek) közül választhatunk. A kiválasztott alakzat kirajzolását a \textit{Draw} jelölőnégyzettel kapcsolhatjuk ki vagy be (alapértelmezésben az összes alakzatot kirajzolja a program). Az utolsó két sorban az adott alakzatot alkotó vertexek, illetve ezekhez tartozó indexek száma látható.

\subsection{Anyagtulajdonságok}

\begin{wrapfigure}{l}{0.5\textwidth}
    \vspace{-23pt}
    \includegraphics[width=0.48\textwidth]{images/pbr_materials.png}
    \vspace{-20pt}
\end{wrapfigure}

A modellhez tartozó .mtl fájlból olvasott fizikai alapú anyagok a legfelső választógombra kattintva érhetők el, ezek valamelyikére kattintva az ablak tartalma automatikusan frissül.

Amennyiben a modellhez nem volt mellékelve anyagleíró fájl, vagy csak szeretnénk kipróbálni valamilyen valódi anyagot, a \textit{Use predefined material}-t bekapcsolva elérhetővé válik az alatta levő lista, illetve ekkor a modell is a választott előre definiált anyag kinézetével jelenik meg. A megjelenítő három beépített anyagot tartalmaz: arany, ezüst és réz.

Az első anyagtulajdonság a diffúz szín, ami megadható egyetlen színként vagy textúrával is. Ez utóbbi csak akkor elérhető, ha az anyagleíróban volt erre vonatkozó bejegyzés. A hozzátartozó kapcsolóval a beállított szín és (ha elérhető,) a betöltött textúra között válthatunk. A következő tulajdonság a normáltextúra. Ezt is az anyagleíróból olvassa a program, a mellett található jelölőnégyzettel pedig a normáltérképezés kapcsolhatjuk be/ki.

Az utolsó két tulajdonság a fizikai alapú megjelenítéshez tartozik. Az első a rücskösség (\textit{roughness}), amely a felület durvaságát szimbolizálja egy 0 és 1 közé eső értékkel. A tökéletesen sima felületeknél ez az érték 0, míg a legdurvább felületek esetén 1. A diffúz színhez hasonlóan ez az érték is olvasható textúrából, ehhez egy egyedi \textit{map\_Roughness} kulcsot kell az anyagleíró fájlhoz hozzáadnunk. Az utolsó tulajdonság a fémesség (\textit{metalness}), amely a rücskösséghez nagyon hasonlóan működik, csak ennél az adható meg, hogy az anyag szigetelő-e vagy sem. Szigorúan fizikai értelemben ennek értéke csak és kizárólag 0 vagy 1 lehetne, mivel nem tudunk részben szigetelő, részben fémes anyagról. Fontos tudni, hogy ezek az anyagtulajdonságok mindig egy adott felület legkülső rétegét írják le: egy festett fémfelületnél tehát ez az érték 0 lesz, hiszen hiába fém az alapja, a fény a festékről verődik vissza, az pedig fizikailag szigetelő. A fémességet meghatározó textúrát szintén egy egyedi, \textit{map\_Metalness} kulccsal tárolhatjuk el.

\begin{wrapfigure}{l}{0.5\textwidth}
    \vspace{-23pt}
    \includegraphics[width=0.48\textwidth]{images/bp_materials.png}
    \vspace{-20pt}
\end{wrapfigure}

Ha a modell információs ablakon Blinn-Phong árnyalásra váltunk, akkor az anyagtulajdonságok automatikusan átváltanak Blinn-Phong nézetre. Az \textit{Ambient color} színválasztó segítségével az ambiens, azaz a jelenetben alapból jelen levő sugárzás színét adhatjuk meg. Fizikai alapú megjelenítés esetén erre nincs szükség, hiszen a kép alapú megvilágítás miatt ez jelenettől függetlenül, automatikusan kiszámolódik.

A diffúz és a normál textúrák megegyeznek a fizikai alapú anyagoknál leírtakkal. A spekuláris szín vagy textúra segítségével a spekuláris csillanások helye, intenzitása és színe adható meg. A \textit{Shininess} (,,ragyogóság'') paraméterrel a felület rücskössége szimulálható: minél nagyobb, annál simább a felület, ezáltal annál kisebb és koncentráltabb lesz a spekuláris csillanás.

\clearpage

\subsection{Megvilágítás beállításai}

\begin{wrapfigure}{l}{0.3\textwidth}
    \vspace{-23pt}
    \includegraphics[width=0.28\textwidth]{images/lighting_settings.png}
    \vspace{-20pt}
\end{wrapfigure}

A megvilágítási beállításoknál lehetőségünk van a megjelenítő által támogatott megvilágítási módok (kép alapú, pont és zseblámpa fény) tulajdonságainak megváltoztatására. A \textit{Use IBL} jelölőnégyzet segítségével be- és kikapcsolhatjuk a kép alapú megvilágítást. Szemléltetés céljából egy pont és egy zseblámpa fényt támogat a program. Mivel a megjelenítő az ún. \textit{forward rendering} technikát használja, ezért több ilyen, dinamikus fényforrás használatára nem alkalmas, mert a fényforrások számával arányosan romlana a kirajzolás sebessége.

A pont fényt a \textit{Use} jelölőnégyzettel kapcsolhatjuk fel vagy le. A \textit{Move} kapcsolóval a pont fény automatikus, képzeletbeli gömbön történő mozgatását tudjuk engedélyezni vagy letiltani. Ezen kívül a kisugárzott fény színét és a fényforrás teljesítményét is változtathatjuk, ez utóbbit 0 és 200 Watt között.

A zseblámpa fény paraméterezése nagyrészt megegyezik a pont fényével. A két új paraméter (\textit{Inner/outer cone angle}) a zseblámpa fénykúpjának belső és külső félszögét határozza meg. A külső félszög adja meg, hogy meddig látható a zseblámpa fénye, a belső félszög pedig arra szolgál, hogy segítségével a valódi zseblámpáknál tapasztalható elhalványodást szimuláljuk a belső és külső félszög között. A zseblámpa pozíciója mindig a nézeti pozíció, iránya pedig a nézeti irány, azaz a zseblámpa fénye mindig a képernyő ,,közepéből'' világít a modellre, és a modell kicsinyítése/nagyítása esetén változik a kivetített fénykör mérete, hiszen ilyenkor tulajdonképpen a kamera modelltől vett távolságát változtatjuk.
