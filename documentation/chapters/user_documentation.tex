
\section{Követelmények}

A mellékelt bináris 64 bites Windows operációs rendszerhez készült, azonban a forráskód a három legnagyobb asztali operációs rendszer (Windows, Linux, OS X) bármelyikére lefordítható, 32 és 64 bites változatban is, ennek menetéről bővebben a következő fejezet ad felvilágosítást.

Mivel a megjelenítő OpenGL alapú, ezért a futtató számítógépen elérhetőnek kell lennie az OpenGL 3.3-as verziójának. Ezt a verziót mind az integrált, mind a dedikált videokártyák évekre visszamenőleg támogatják. Ha nem vagyunk biztosak a támogatott verzióban, akkor a grafikus meghajtóprogram vezérlőpultján valószínűleg megtalálható ez az információ.

\section{A program üzembe helyezése}

Ha 64 bites Windows rendszert használunk, akkor a mellékelt, előre lefordított bináris használható. Ha viszont más operációs rendszeren dolgozunk, vagy tovább szeretnénk fejleszteni a programot, akkor azt az adott platformra le kell fordítsuk. Ennek minimális feltétele a Git verziókövető rendszer és a CMake keresztplatformos buildrendszer megléte a számítógépen.

Az első fordítás előkészítése:

\begin{enumerate}[noitemsep]
\item nyissunk meg egy parancssort (terminált) a program gyökérkönyvtárában
\item 
\end{enumerate}

A fordítás menete:

\section{A felhasználói felület áttekintése}

% ábra 