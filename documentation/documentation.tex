
% Preamble

\documentclass[12pt,oneside]{book}

\usepackage{amsmath} % Advanced math typesetting
\usepackage[utf8]{inputenc} % Unicode support (Umlauts etc.)
\usepackage[explicit]{titlesec}
\usepackage{t1enc}
\usepackage[hungarian]{babel} % Change hyphenation rules
%\usepackage{hyperref} % Add a link to your document
\usepackage{graphicx} % Add pictures to your document
\usepackage{listings} % Source code formatting and highlighting
\usepackage[a4paper, top=2.5cm, bottom=2.5cm, left=3.5cm, right=2.5cm]{geometry}
%\usepackage{sidecap}
\usepackage{wrapfig}

\usepackage{setspace}
\onehalfspacing

%\linespread{1.3}

\begin{document}

% Belső fedőlap

\begin{titlepage}

\noindent
\begin{minipage}{4cm}
    \includegraphics[width=3cm]{cimer_nagy_szines.png}
\end{minipage}
\hfill
\begin{minipage}[c]{8cm}
Eötvös Loránd Tudományegyetem\\
Informatikai Kar\\
Algoritmusok és Alkalmazásaik Tanszék
\end{minipage}

\vspace{5cm}

\begin{center}
\huge
Fizikai alapú 3D megjelenítés valósidőben
\end{center}

\vspace{4cm}

\noindent
\begin{minipage}[t]{7cm}
\flushleft
Valasek Gábor\\
tanársegéd
\end{minipage}
\hfill
\begin{minipage}[t]{5.2cm}
%\flushright
Bölöny Zsolt\\
programtervező informatikus
\end{minipage}

\vspace{1cm}

\noindent
\begin{minipage}[t]{7cm}
\flushleft
Dr. Magdics Milán\\
adjunktus
\end{minipage}

\begin{center}
Budapest, 2016
\end{center}

\end{titlepage}

% Tartalomjegyzék

\tableofcontents

% Bevezetés

\chapter{Bevezetés}
A számítógépes grafikával foglalkozó kutatókat, fejlesztőket napjainkban is foglalkoztatja a kérdés, hogyan lehet a számítógép segítségével minél életszerűbb, \textit{fotorealisztikus} megjelenítést létrehozni. A különféle sugárkövetéses technikákkal már régóta képesek vagyunk rendkívül élethű képek létrehozására, de ezek a módszerek nem minden esetben jelentenek megoldást, hiszen a rendelkezésre álló számítási kapacitástól függően a létrehozás folyamata akár napokig is eltarthat. Míg egy animációs film esetében van lehetőség ezt kivárni, addig a valós idejű megjelenítésben nem áll rendelkezésre ennyi idő: a szinte folyton változó, interaktív 3D-s környezetet a másodperc törtrésze alatt kell a képernyőre vetítenünk.

A grafikus kártyák teljesítménye a megjelenésük óta folyamatosan, gyors ütemben fejlődik. Évről évre újabb és újabb technikák jelennek meg, amelyek igyekeznek kihasználni ezt a folyton növekvő teljesítményt annak érdekében, hogy minél látványosabb, minél valószerűbb legyen a virtuálisan létrehozott 3D-s tartalmak megjelenítése. A fizikai alapú megjelenítés is csupán egy, bár annál jelentősebb módszer ezek közül: nem csak a korábban mindenki által használt Blinn-Phong közelítést cseréli le egy, az anyagok valós, fizikai tulajdonságait figyelembe vevő új módszerre, hanem a megjelenítés bemeneteként szolgáló anyagmodellek előállításának folyamatát is átformálja.

A fizikai alapú megjelenítés (hivatalosan \textit{physically based rendering}, a továbbiakban PBR) legfontosabb tulajdonsága az, hogy az anyagok ill. fények mérhető fizikai tulajdonságait veszi alapul, és az adott pontban a kamera által érzékelt fény színét és intenzitását igyekszik ismert, valós fizikai képletek közelítéseinek segítségével meghatározni. Habár a PBR alapelvei és összefüggései létrejötte óta változatlanok, a felhasznált különféle egyenletek minél pontosabb és mindeközben minél kevésbé számításigényes közelítése a mai napig aktív kutatási terület.

A fizikai alapú megjelenítés előnyeinek bemutatására egy egyszerű modellek betöltésére képes megjelenítőt hoztam létre, amelynek segítségével akár egy témában kevésbé járatos felhasználó számára is láthatóvá válnak a módszer előnyei. A program megalkotása komoly kihívást jelentett, hiszen nem csak a CPU-t, hanem a GPU-t is programoznom kellett árnyalók (\textit{shaderek}) segítségével. A felhasznált technológia újdonsága, az implementáció komplexitása és a várható látványos végeredmény miatt esett választásom erre a feladatra.

% Felhasználói dokumentáció

\chapter{Felhasználói dokumentáció}
\section{kukac}
nyehehe

\chapter{Fejlesztői dokumentáció}
semmi

\end{document} 