
% Preamble

\documentclass[12pt,oneside]{book}

\usepackage{amsmath} % Advanced math typesetting
\usepackage[utf8]{inputenc} % Unicode support (Umlauts etc.)
\usepackage[explicit]{titlesec}
\usepackage{t1enc}
\usepackage[hungarian]{babel} % Change hyphenation rules
%\usepackage{hyperref} % Add a link to your document
\usepackage{graphicx} % Add pictures to your document
\usepackage{listings} % Source code formatting and highlighting
\usepackage[a4paper, top=2.5cm, bottom=2.5cm, left=3.5cm, right=2.5cm]{geometry}
%\usepackage{sidecap}
\usepackage{wrapfig}
\usepackage{enumitem}

\usepackage{setspace}
\onehalfspacing

%\linespread{1.3}

\begin{document}

% Belső fedőlap

\begin{titlepage}

\noindent
\begin{minipage}{4cm}
    \includegraphics[width=3cm]{cimer_nagy_szines.png}
\end{minipage}
\hfill
\begin{minipage}[c]{8cm}
Eötvös Loránd Tudományegyetem\\
Informatikai Kar\\
Algoritmusok és Alkalmazásaik Tanszék
\end{minipage}

\vspace{5cm}

\begin{center}
\huge
Fizikai alapú 3D megjelenítés valósidőben
\end{center}

\vspace{4cm}

\noindent
\begin{minipage}[t]{7cm}
\flushleft
Valasek Gábor\\
tanársegéd
\end{minipage}
\hfill
\begin{minipage}[t]{5.2cm}
%\flushright
Bölöny Zsolt\\
programtervező informatikus
\end{minipage}

\vspace{1cm}

\noindent
\begin{minipage}[t]{7cm}
\flushleft
Dr. Magdics Milán\\
adjunktus
\end{minipage}

\begin{center}
Budapest, 2016
\end{center}

\end{titlepage}

% Tartalomjegyzék

\tableofcontents

% Bevezetés

\chapter{Bevezetés}
A számítógépes grafikával foglalkozó kutatókat, fejlesztőket napjainkban is foglalkoztatja a kérdés, hogyan lehet a számítógép segítségével minél életszerűbb, \textit{fotorealisztikus} megjelenítést létrehozni. A különféle sugárkövetéses technikákkal már régóta képesek vagyunk rendkívül élethű képek létrehozására, de ezek a módszerek nem minden esetben jelentenek megoldást, hiszen a rendelkezésre álló számítási kapacitástól függően a létrehozás folyamata akár napokig is eltarthat. Míg egy animációs film esetében van lehetőség ezt kivárni, addig a valós idejű megjelenítésben nem áll rendelkezésre ennyi idő: a szinte folyton változó, interaktív 3D-s környezetet a másodperc törtrésze alatt kell a képernyőre vetítenünk.

A grafikus kártyák teljesítménye a megjelenésük óta folyamatosan, gyors ütemben fejlődik. Évről évre újabb és újabb technikák jelennek meg, amelyek igyekeznek kihasználni ezt a folyton növekvő teljesítményt annak érdekében, hogy minél látványosabb, minél valószerűbb legyen a virtuálisan létrehozott 3D-s tartalmak megjelenítése. A fizikai alapú megjelenítés is csupán egy, bár annál jelentősebb módszer ezek közül: nem csak a korábban mindenki által használt Blinn-Phong közelítést cseréli le egy, az anyagok valós, fizikai tulajdonságait figyelembe vevő új módszerre, hanem a megjelenítés bemeneteként szolgáló anyagmodellek előállításának folyamatát is átformálja.

A fizikai alapú megjelenítés (hivatalosan \textit{physically based rendering}, a továbbiakban PBR) legfontosabb tulajdonsága az, hogy az anyagok ill. fények mérhető fizikai tulajdonságait veszi alapul, és az adott pontban a kamera által érzékelt fény színét és intenzitását igyekszik ismert, valós fizikai képletek közelítéseinek segítségével meghatározni. Habár a PBR alapelvei és összefüggései létrejötte óta változatlanok, a felhasznált különféle egyenletek minél pontosabb és mindeközben minél kevésbé számításigényes közelítése a mai napig aktív kutatási terület.

A fizikai alapú megjelenítés előnyeinek bemutatására egy egyszerű modellek betöltésére képes megjelenítőt hoztam létre, amelynek segítségével akár egy témában kevésbé járatos felhasználó számára is láthatóvá válnak a módszer előnyei. A program megalkotása komoly kihívást jelentett, hiszen nem csak a CPU-t, hanem a GPU-t is programoznom kellett árnyalók (\textit{shaderek}) segítségével. A felhasznált technológia újdonsága, az implementáció komplexitása és a várható látványos végeredmény miatt esett választásom erre a feladatra.

% Felhasználói dokumentáció

\chapter{Felhasználói dokumentáció}
\section{kukac}
nyehehe

% Fejlesztői dokumentáció

\chapter{Fejlesztői dokumentáció}

\section{A 3D-s megjelenítés elméleti háttere}

\subsection{A megjelenítési egyenlet}

Tetszőleges térbeli jelenet leképezéséhez alapvetően három dolog szükséges:

\begin{itemize}[noitemsep]
\item a jelenet geometriájának leírása,
\item egy pont a térben, ahonnan "nézzük" a jelenetet, továbbá
\item legalább egy fényforrás, ill. annak pozíciója.
\end{itemize}

Valósidejű grafikában a geometriák leírásához háromszöghálókat használunk, mivel a GPU-k térbeli háromszögeken dolgoznak. Domború felületek leírásához ezen háromszögháló felbontását növeljük addig, amíg a végeredmény szempontjából elfogadható közelítést kapunk. Ezt a folyamatot tesszelációnak hívjuk. Az "elfogadhatóság" teljesen szubjektív tulajdonság, így az egyetlen objektív behatároló tényező a grafikus hardver teljesítménye, amelynek folyamatos fejlődése ezen a téren remekül illusztrálható.

%TODO: régi baltával faragott 3d modell kép vs. modern

Önmagukban a háromszögeket meghatározó térbeli pozíciók még nem elégségesek. Ismernünk kell a felület tetszőleges pontjának irányultságát, amelyet egy, a felületből "kifelé" álló, normalizált vektorral írunk le és (felületi) normálvektornak hívunk. Jelölése: n.
%TODO: ezt inkább a gyakorlati részhez, a vertex definíciójával együtt?
Ezeket legegyszerűbben a háromszögeket alkotó pontokkal együtt tárolhatjuk, majd ezeket menet közben interpolálva kaphatjuk meg a háromszög által meghatározott felület tetszőleges pontján vett normálvektort.

\subsection{A megjelenítési egyenlet}

A számítógépes grafika egyik alaptételének tekinthető megjelenítési egyenletet (\textit{rendering equation}) David Immel et al. és James Kajiya írta le 1986-ban. Az egyenlet segítségével meghatározható a felület egy adott pontját elhagyó sugárzás a felület által kibocsátott ill. visszavert sugárzás összegének geometriai optika alapú közelítésével:

\[
L_0(\mathbf{x},\mathbf{v}) = L_e(\mathbf{x},\mathbf{v}) + \int_\Omega f_r(\mathbf{x},\mathbf{l},\mathbf{v}) L_i(\mathbf{x},\mathbf{l}) (-\mathbf{l} \cdot \mathbf{n})\,\mathrm{d}\mathbf{l}
\]

ahol

\begin{itemize}[noitemsep]
\item \(\mathbf{x}\) a felület adott pontja
\item \(\mathbf{v}\) az x vektorból a nézeti pozícióba mutató vektor (nézeti vektor)
\item \(\mathbf{n}\) a korábban bevezetett felületi normálvektor
\item \(\mathbf{l}\) a beérkező fény negatív iránya
\item \(\Omega\) dd
\item \(L_e\) a felület által kibocsátott fény az adott pontban
\item \(f_r\) 
\item \(L_i\)
\end{itemize}

\textit{Megjegyzés: az eredeti egyenlet figyelembe veszi még az időt és a fény hullámhosszát is. Az egyenlet paraméterei az idő előrehaladtával ritkán változnak, és ebben az esetben is előre kiszámolhatóak, ezért az időt állandónak tekinthetjük, amely így kiesik az egyenletből. A gyakorlatban RGB színtérben dolgozunk, és ennek 3 összetevőjét külön-külön számoljuk, így a hullámhossz paraméter is elhagyható.}

Számítógépes grafikában a kimenetünk színértékek kétdimenziós mátrixa, jellemzően a képernyő pixelei. Ahhoz tehát, hogy kimenetet állítsunk elő, ezen

\end{document} 