
% Preamble

\documentclass[12pt,oneside]{book}

\usepackage{amsmath} % Advanced math typesetting
\usepackage[utf8]{inputenc} % Unicode support (Umlauts etc.)
\usepackage[explicit]{titlesec}
\usepackage{t1enc}
\usepackage[hungarian]{babel} % Change hyphenation rules
%\usepackage{hyperref} % Add a link to your document
\usepackage{graphicx} % Add pictures to your document
\usepackage{listings} % Source code formatting and highlighting
\usepackage[a4paper, top=2.5cm, bottom=2.5cm, left=3.5cm, right=2.5cm]{geometry}
%\usepackage{sidecap}
\usepackage{wrapfig}
\usepackage{enumitem}

\usepackage{setspace}
\onehalfspacing

%\linespread{1.3}

\begin{document}

% Belső fedőlap

\begin{titlepage}

\noindent
\begin{minipage}{4cm}
    \includegraphics[width=3cm]{cimer_nagy_szines.png}
\end{minipage}
\hfill
\begin{minipage}[c]{8cm}
Eötvös Loránd Tudományegyetem\\
Informatikai Kar\\
Algoritmusok és Alkalmazásaik Tanszék
\end{minipage}

\vspace{5cm}

\begin{center}
\huge
Fizikai alapú 3D megjelenítés valósidőben
\end{center}

\vspace{4cm}

\noindent
\begin{minipage}[t]{7cm}
\flushleft
Valasek Gábor\\
tanársegéd
\end{minipage}
\hfill
\begin{minipage}[t]{5.2cm}
%\flushright
Bölöny Zsolt\\
programtervező informatikus
\end{minipage}

\vspace{1cm}

\noindent
\begin{minipage}[t]{7cm}
\flushleft
Dr. Magdics Milán\\
adjunktus
\end{minipage}

\begin{center}
Budapest, 2016
\end{center}

\end{titlepage}

% Tartalomjegyzék

\tableofcontents

% Bevezetés

\chapter{Bevezetés}
A számítógépes grafikával foglalkozó kutatókat, fejlesztőket napjainkban is foglalkoztatja a kérdés, hogyan lehet a számítógép segítségével minél életszerűbb, \textit{fotorealisztikus} megjelenítést létrehozni. A különféle sugárkövetéses technikákkal már régóta képesek vagyunk rendkívül élethű képek létrehozására, de ezek a módszerek nem minden esetben jelentenek megoldást, hiszen a rendelkezésre álló számítási kapacitástól függően a létrehozás folyamata akár napokig is eltarthat. Míg egy animációs film esetében van lehetőség ezt kivárni, addig a valós idejű megjelenítésben nem áll rendelkezésre ennyi idő: a szinte folyton változó, interaktív 3D-s környezetet a másodperc törtrésze alatt kell a képernyőre vetítenünk.

A grafikus kártyák teljesítménye a megjelenésük óta folyamatosan, gyors ütemben fejlődik. Évről évre újabb és újabb technikák jelennek meg, amelyek igyekeznek kihasználni ezt a folyton növekvő teljesítményt annak érdekében, hogy minél látványosabb, minél valószerűbb legyen a virtuálisan létrehozott 3D-s tartalmak megjelenítése. A fizikai alapú megjelenítés is csupán egy, bár annál jelentősebb módszer ezek közül: nem csak a korábban mindenki által használt Blinn-Phong közelítést cseréli le egy, az anyagok valós, fizikai tulajdonságait figyelembe vevő új módszerre, hanem a megjelenítés bemeneteként szolgáló anyagmodellek előállításának folyamatát is átformálja.

A fizikai alapú megjelenítés (hivatalosan \textit{physically based rendering}, a továbbiakban PBR) legfontosabb tulajdonsága az, hogy az anyagok ill. fények mérhető fizikai tulajdonságait veszi alapul, és az adott pontban a kamera által érzékelt fény színét és intenzitását igyekszik ismert, valós fizikai képletek közelítéseinek segítségével meghatározni. Habár a PBR alapelvei és összefüggései létrejötte óta változatlanok, a felhasznált különféle egyenletek minél pontosabb és mindeközben minél kevésbé számításigényes közelítése a mai napig aktív kutatási terület.

A fizikai alapú megjelenítés előnyeinek bemutatására egy egyszerű modellek betöltésére képes megjelenítőt hoztam létre, amelynek segítségével akár egy témában kevésbé járatos felhasználó számára is láthatóvá válnak a módszer előnyei. A program megalkotása komoly kihívást jelentett, hiszen nem csak a CPU-t, hanem a GPU-t is programoznom kellett árnyalók (\textit{shaderek}) segítségével. A felhasznált technológia újdonsága, az implementáció komplexitása és a várható látványos végeredmény miatt esett választásom erre a feladatra.

% Felhasználói dokumentáció

\chapter{Felhasználói dokumentáció}
\section{A program üzembe helyezése}
\section{A felhasználói felület áttekintése}

% Fejlesztői dokumentáció

\chapter{Fejlesztői dokumentáció}

\section{A 3D-s megjelenítés elméleti háttere}

\subsection{Bevezetés}

Tetszőleges térbeli jelenet leképezéséhez alapvetően három dolog szükséges:

\begin{itemize}[noitemsep]
\item a jelenet geometriájának leírása,
\item egy pont a térben, ahonnan "nézzük" a jelenetet, továbbá
\item legalább egy fényforrás, ill. annak pozíciója.
\end{itemize}

Valósidejű grafikában a geometriák leírásához háromszöghálókat használunk, mivel a GPU-k térbeli háromszögeken dolgoznak. Domború felületek leírásához ezen háromszögháló felbontását növeljük addig, amíg a végeredmény szempontjából elfogadható közelítést kapunk. Ezt a folyamatot tesszelációnak hívjuk. Az "elfogadhatóság" teljesen szubjektív tulajdonság, így az egyetlen objektív behatároló tényező a grafikus hardver teljesítménye, amelynek folyamatos fejlődése ezen a téren remekül illusztrálható.

%TODO: régi baltával faragott 3d modell kép vs. modern

Önmagukban a háromszögeket meghatározó térbeli pozíciók még nem elégségesek. Ismernünk kell a felület tetszőleges pontjának irányultságát, amelyet egy, a felületből "kifelé" álló, egység hosszúságú vektorral írunk le és felületi normálvektornak hívunk. Jelölése: \(\mathbf{n}\).
%TODO: ezt inkább a gyakorlati részhez, a vertex definíciójával együtt?
Ezeket legegyszerűbben a háromszögeket alkotó pontokkal együtt tárolhatjuk, majd ezeket menet közben interpolálva kaphatjuk meg a háromszög által meghatározott felület tetszőleges pontján vett normálvektort.

\subsection{Az árnyalási egyenlet}

A számítógépes grafika egyik alaptételének tekinthető árnyalási egyenletet (\textit{rendering equation}) David Immel et al. és James Kajiya írta le 1986-ban. Az egyenlet segítségével meghatározható a felület egy adott pontját elhagyó sugárzás a felület által kibocsátott ill. visszavert sugárzás összegének geometriai optika alapú közelítésével:

\[
L_0(\mathbf{x},\mathbf{v}) = L_e(\mathbf{x},\mathbf{v}) + \int_\Omega f_r(\mathbf{x},\mathbf{l},\mathbf{v}) L_i(\mathbf{x},\mathbf{l}) (-\mathbf{l} \cdot \mathbf{n})\,\mathrm{d}\mathbf{l}
\]

\noindent
ahol a paraméterek:

\begin{itemize}[noitemsep]
\item \(\mathbf{x}\) a felület egy pontja,
\item \(\mathbf{v}\) a felület egy pontjából a nézeti pozícióba mutató normálvektor (nézeti vektor),
\item \(\mathbf{n}\) a korábban bevezetett felületi normálvektor,
\item \(\mathbf{l}\) pedig a felület egy pontjából a fényforrás felé mutató normálvektor.
\end{itemize}

\noindent
\textit{Megjegyzés: az eredeti egyenlet figyelembe veszi még az időt és a fény hullámhosszát is. Az egyenlet paraméterei az idő előrehaladtával ritkán változnak, és ebben az esetben is előre kiszámolhatóak, ezért gyakorlati alkalmazás esetén az időt állandónak tekinthetjük, amely így kiesik az egyenletből. Mivel számítógépes grafikánál RGB színtérben dolgozunk, és ennek 3 összetevőjét külön-külön számolhatjuk, így a hullámhossz paraméter is elhagyható. A továbbiakban ezért fény alatt a fény színét és erősségét értjük.}

A könnyebb érthetőség kedvéért bontsuk részekre az egyenletet. A bal oldalon szereplő \(L_0)\) az eredmény: a felület egy pontjából érkező fényt határozza meg a nézeti vektor függvényében, később ez lesz a ténylegesen megjelenített színérték. A jobb oldal első tagja, \(L_e\) a felület egy pontjából kisugárzott fényt adja meg. Ez az érték a gyakorlatban jellemzően 0, mert kevés emisszív anyag létezik, de pl. a fényforrások megjelenítéséhez szükség van rá. A következő tag az integrál, amelyben az \(\Omega\) a felületi normálvektor körül vett félgömböt jelenti, ami az összes lehetséges \(\mathbf{l}\) vektort tartalmazza, amelyen integrálni kell a tartalmazott függvényeket. Mint később látni fogjuk, az integrálás lesz az egyik sarkalatos pontja az egyenlet gyakorlati alkalmazásának, ugyanis az integrálnak nem minden fényforrás esetén van analitikus megoldása - ilyenkor közelítést fogunk alkalmazni. \(f_r\) az ún. kétirányú visszaverődési eloszlásfüggvény (bi-directional reflection distribution function, röviden BRDF), amely a nézeti irányba visszavert fény és a fényforrás felől besugárzott fény arányát adja meg. A BRDF nagy előnye, hogy fizikailag mérhető, és az interneten található MERL adatbázis 100 különböző anyag visszaverődési függvényét tartalmazza. Tulajdonságai:

\begin{itemize}[noitemsep]
\item pozitív: \(f_r(\mathbf{x},\mathbf{l},\mathbf{v}) \geq 0\)
\item szimmetrikus: \(f_r(\mathbf{x},\mathbf{l},\mathbf{v}) = f_r(\mathbf{x},\mathbf{v},\mathbf{l})\)
\item teljesíti az energiamegmaradás törvényét: \(\int_\Omega f_r(\mathbf{x},\mathbf{l},\mathbf{v}) (-\mathbf{l} \cdot \mathbf{n})\,\mathrm{d}\mathbf{l} \leq 1\)
\end{itemize}

Az \(L_i\) függvény a felület adott pontjára \(-\mathbf{l}\) irányból beérkező fényt adja meg. Fontos megjegyezni, hogy ez a fény nem csak direkt, hanem indirekt forrásból is érkezhet (pl. már valahol tükröződött fény, ld. globális/kép alapú megvilágítás). Az utolsó tag, \(-\mathbf{l} \cdot \mathbf{n}\) a beérkező fény iránya és a felületi normálvektor által bezárt szög koszinusza alapján csökkenti a kisugárzott fény erősségét.

\subsection{Fizikai alapú megjelenítés}

\section{Gyakorlati megvalósítás}
\subsection{A felhasznált nyílt forráskódú könyvtárak bemutatása}
\subsection{A megjelenítő felépítése}

\subsection{Nagy dinamikatartományú megjelenítés}

A jelenleg elterjedt kijelzők túlnyomó többsége 24 bites RGB bemenet alapján dolgozik. Ez azt jelenti, hogy a képernyőn látható színek vörös, zöld és kék komponensekből állnak, ahol minden komponensre \(2^8\) bit jut, azaz 256 különböző értéket vehetnek fel. Elméletben tehát \(3 \cdot 2^8\), azaz körülbelül 16,7 millió különböző színt tudunk megjeleníteni. A valóságban ténylegesen megjelenített színmennyiség a kijelzők különböző fizikai paramétereitől függően változik, de az emberi szem csupán körülbelül 10 millió színt képes megkülönböztetni egymástól~\cite{judd1975color}, így ebben a tekintetben a technológia az emberi érzékelés előtt áll.

Van azonban egy másik nagyon fontos, megjelenítőt és érzékelőt egyaránt jellemző tulajdonság, a kontrasztarány, amit a legvilágosabb (fehér) és a legsötétebb (fekete) megjeleníteni/érzékelni képes szín közötti arányként definiálunk. Míg egy modern LCD kijelző a 24 bites színtér által meghatározott 256:1 bemenő kontrasztarányból kb. 1000:1 kimenő arányt tud megjeleníteni, addig az emberi szem különböző fényviszonyokhoz való rendkívül jó alkalmazkodóképességének köszönhetően ennél jóval nagyobb, kb. 1000:1 - 15000:1 kontrasztarány érzékelésére is képes. (hivatkozás?)

A nagy dinamikatartomány (\textit{high-dynamic-range, HDR}) fogalma először a fényképészet területén jelent meg. A probléma az volt - illetve mind a mai napig az, hogy a fényképezőgépek felépítéséből adódóan egyetlen expozícióval nem lehet olyan képet készíteni, amely visszaadja az emberi szem által érzékelhető dinamikatartományt. A fényképészek ezt úgy oldották meg, hogy az expozíciós időt - és ezzel a bejövő fénymennyiséget - változtatva több képet készítettek ugyanarról a jelenetről, majd egy végső lépésben (színleképezés, \textit{tone mapping}) ezeket egy meghatározott metodika alapján egy képpé kombinálták. A fényképészet a dinamikatartományt ún. expozíciós értékkel (\textit{exposure value, EV}) méri, ahol EV eggyel való növekedése a beérkező fénymennyiség megkétszereződését jelenti.

Ahhoz tehát, hogy az általunk a valóságban érzékelhető fényerősség különbségeket érzékeltetni tudjuk, először is el kell szakadjunk az ebben minket korlátozó 24 bites színtértől. A modern grafikus kártyák már hardveresen támogatják a lebegőpontos (komponensenként 16 vagy 32 bites) puffereket, így ezeket használhatjuk HDR megjelenítésre. A fényképészettel ellentétben továbbra is csak egy leképezési lépésre van szükségünk, amely során egy ilyen lebegőpontos pufferbe számolunk, amelyben a korábbi maximális 255-nél fényerősségtől függően jóval nagyobb értékek is szerepelhetnek. A kijelzők azonban továbbra is a 24 bites, alacsony dinamikatartományú RGB színtérben várják a bemenetet, ezért HDR megjelenítésnél is szükség van egy színleképezési lépésre, amely során egy ún. színleképezési operátor segítségével a pufferben szereplő értékeket ismét "beszorítjuk" a \([0-1]\) tartományba. Színleképezés során az alapvető cél az eredeti, HDR kép lokális kontrasztarányainak minél jobb megtartása. Több ilyen operátor létezik, az egyik legegyszerűbb ezek közül a Reinhard operátor~\cite{reinhard2002photographic}:

\[
L_ldr = { L_hdr(x, y) \over 1 + L_hdr(x, y) }
\]

ahol ...

Adaptáció?

\subsection{Kép alapú megvilágítás}
\subsection{Pont- és zseblámpa fények}

\section{Tesztelés}

\bibliography{documentation}{}
\bibliographystyle{plain}

\end{document} 